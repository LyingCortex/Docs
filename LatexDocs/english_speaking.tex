\documentclass[a4paper, 11pt]{article}

%%%%%% 导入包 %%%%%%
\usepackage{CJKutf8}
\usepackage{graphicx}
\usepackage[unicode]{hyperref}
\usepackage{xcolor}
\usepackage{cite}
\usepackage{indentfirst}

%%%% 正文开始 %%%%
\begin{document}
\begin{CJK}{UTF8}{gbsn}

%%%% 定理类环境的定义 %%%%
\newtheorem{example}{例}             % 整体编号
\newtheorem{algorithm}{算法}
\newtheorem{theorem}{定理}[section]  % 按 section 编号
\newtheorem{definition}{定义}
\newtheorem{axiom}{公理}
\newtheorem{property}{性质}
\newtheorem{proposition}{命题}
\newtheorem{lemma}{引理}
\newtheorem{corollary}{推论}
\newtheorem{remark}{注解}
\newtheorem{condition}{条件}
\newtheorem{conclusion}{结论}
\newtheorem{assumption}{假设}

%%%% 重定义 %%%%
\renewcommand{\contentsname}{目录}  % 将Contents改为目录
\renewcommand{\abstractname}{摘要}  % 将Abstract改为摘要
\renewcommand{\refname}{参考文献}   % 将References改为参考文献
\renewcommand{\indexname}{索引}
\renewcommand{\figurename}{图}
\renewcommand{\tablename}{表}
\renewcommand{\appendixname}{附录}
\renewcommand{\algorithm}{算法}







Note-taking is a powerful skill that help you better understand what others'talking,especially when you attendding a meeting or lecture,or just in classroom. To be honest , 

Note-taking is the practice of recording information captured from another source. By taking notes, the writer records the essence of the information, freeing their mind from having to recall everything.[1] Notes are commonly drawn from a transient source, such as an oral discussion at a meeting, or a lecture (notes of a meeting are usually called minutes), in which case the notes may be the only record of the event. Note taking is a form of self-discipline.
he person taking notes must acquire and filter the incoming sources, organize and restructure existing knowledge structures, comprehend and write down their interpretation of the information, and ultimately store and integrate the freshly processed material. The result is a knowledge representation, and a memory storage.[1]

There are many types of non-linear note-taking techniques, including: Clustering,[6] Concept mapping,[7][8] Cornell system,[9] Idea mapping,[10] Instant replays,[11] Ishikawa diagrams,[12] Knowledge maps,[13] Learning maps,[14] Mind mapping,[15] Model maps,[16] Pyramid principle,[17] Semantic networks,[18] SmartWisdom.[19] and Jay's Notes.

The following are details about a few.

Outlines tend to proceed down a page, using headings and bullets to structure information. A common system consists of headings that use Roman numerals, letters of the alphabet, and Arabic numerals at different levels. 
Electronic note-taking methods[edit]
The growing ubiquity of laptops in universities and colleges has led to a rise in electronic note-taking. Many students write their notes in word processors. Online word processor applications are receiving growing attention from students who can forward notes using email, or otherwise make use of collaborative features in these applications and can also download the texts as a file (txt, rtf...) in a local computer.

Online note-taking has created problems for teachers who must balance educational freedom with copyright and intellectual property concerns regarding course content.[citation needed]


\end{CJK}
\end{document}